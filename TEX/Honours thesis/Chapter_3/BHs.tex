\chapter{Black holes}

\section{Schwarzschild's discovery}
Sketch of Schwarzschild's derivation and consequence of singularity. Show that matter from an observing falling into BH never passes event horizon according to outside observer.

The distance between two points in space-time is given by the invariant line element $ds^2$. In Schwarzschild space-time, the line is given by equation (\ref{eq:Schwarzschild line-element}):

\begin{equation}
    ds^2 = - d\tau^2 = - (1-\frac{2 M}{r}) dt^2 +  (1-\frac{2 M}{r})^{-1} dr^2
            + r^2 (d\theta^2+\sin^2\theta d\phi^2)
\label{eq:Schwarzschild line-element}
\end{equation}

\section{Kerr black holes}
Long after Kerr's discovery, other black hole solutions where found. It was shown (no hair theorem) that all black holes can be characterised by a charge and spin (like an electron). 

The metric for a Kerr black hole (spherically symmetric, stationary, black hole of mass M rotating at constant angular momentum J) is given by:

\begin{equation}\label{eq:Kerr metric}
    ds^2 = -\left(1-\frac{2Mr}{\rho^2}\right) dt^2-\frac{4Mar \sin^2{\theta}}{\rho^2} d\phi dt+ \frac{\rho^2}{\Delta} dr^2 + \rho^2 d\theta^2 +\left(r^2+a^2+
    \frac{2Mra^2\sin^2{\theta}}{\rho^2}
    \right)\sin{\theta}^2 d\phi^2
\end{equation}

Where $a:=\frac{J}{M}$, $\rho^2:=r^2+a^2 \cos{\theta}$, and $\Delta:=r^2-2Mr+a^2$ \cite{carroll2019spacetime}.

It is easy to see that setting $a=0$ leaves us with the Schwarzschild solution (the case of zero spin is the same as Schwarzschild solution in equation (\ref{eq:Schwarzschild line-element})), and in the limit that $r \gg a$ we get the weak field metric (in terms of gravitational potential $\Phi(r)=-\frac{G M}{r}$)

\begin{equation}
    ds^2 = - (1+2\Phi(r)) dt^2 +  (1-2\Phi(r)) dr^2
            + r^2 (d\theta^2+\sin^2\theta d\phi^2)
\label{eq:Weak field metric}
\end{equation}

Which as $r\rightarrow \infty$ approaches flat (Minkowski) spacetime

\begin{equation}
    ds^2 = - dt^2 + dx^2 + dy^2 + dz^2
\label{eq:Minkowski metruc}
\end{equation}

Explain Kerr black holes, frame dragging (relate back to weak field limit).

Show plots from Kerr simulation.

\section{Binary black holes}
\subsection{Mergers}

Mention how when two black holes orbit, GW emission results in energy losses, so by conservation of energy gravitational binding energy must decrease leading to inspiral and eventual merger.


\subsection{Analogy with magnetic reconnection}
Looking at magnetic reconnection, it requires a dipole

\subsection{Accreting matter}
. Look at L1 Lagrange point (for matter infalling outside event horizon) or L4 (stable orbits - show simulation)

\subsection{Consequences}
This could cause an ejection, explain how.