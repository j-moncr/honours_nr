% \documentclass[12pt,a4paper]{report}
% \usepackage[letterpaper,top=2cm,bottom=2cm,left=3cm,right=3cm,marginparwidth=1.75cm]{geometry}
% \usepackage{parskip}
% \usepackage{fourier}
% \usepackage{amsmath}
% \usepackage{array}
% \usepackage{graphicx}
% \usepackage{tikz}
% \usepackage{empheq}
% \usepackage[colorlinks=true, allcolors=blue]{hyperref}


% \begin{document}
% \begin{center}
%     {\bfseries\Large
%     Studying the evolution of gravitational fields around compact object binaries - thesis proposal\\}
% \vskip .5cm  
% Jordan Moncrieff\\
% \vskip .5cm
% \textbf{Supervisors:} 
%     Bruce Gendre and Fiona Panther
% \end{center}

\chapter{Thesis proposal}
\section{Aims and significance}

The aim of the thesis is to study the gravitational field within orbiting compact binary objects near the point of merger. The primary objective is to explore a poorly studied  phenomenon, gravitational reconnection. This concept is analogous to its magnetic counterpart, which leads to phenomena such as solar flares and coronal mass ejections. Similarly, gravitational reconnection could result in significant mass ejection or acceleration in the context of merging compact binary objects. By investigating the conditions under which gravitational reconnection can occur, the project aims to gain a deeper understanding of the astrophysical processes involved in the final stages of compact binary mergers. If gravitational reconnection is found to occur in this context, this could have important implications for our understanding of the formation and evolution of astrophysical jets in the universe.

The potential discovery of gravitational reconnection leading to jet formation would imply an electromagnetic counterpart to black hole mergers, much like the post-merger electromagnetic counterparts of neutron stars. This would open up exciting opportunities for further research into the behavior of compact binary mergers and the physics of extreme astrophysical environments. Additionally, this would inform future efforts to perform electromagnetic follow-ups after gravitational wave detections between binary black holes. By combining observations of gravitational waves with electromagnetic radiation at different wavelengths, we can obtain a more complete picture of the physical processes involved in the most extreme environments in the universe, and gain insights into the fundamental laws that govern the behavior of matter and energy on a cosmic scale. Furthermore, these investigations could reveal exotic shapes of the event horizon during these extreme events, providing a new perspective on the dynamical nature of spacetime in the vicinity of merging compact objects.

\section{Status (literature review)}


Since Einstein published his general theory of relativity, it has become evident that gravitation and electromagnetism are intricately intertwined. In the weak field limit of general relativity, the Einstein field equations, governing gravity, simplify to the gravitomagnetic equations, which bear a striking resemblance to Maxwell's equations. Consequently, this remarkable similarity between gravity and electromagnetism has given rise to various analogies between gravitational and electromagnetic phenomena. 

In recent years, the field of astronomy has witnessed a significant transformation, empowering researchers to explore general relativistic phenomena with unprecedented precision. Particularly, multimessenger astronomy has emerged as a powerful approach, combining data from diverse detectors such as gravitational wave detectors, telescopes operating across different wavelengths, and neutrino detectors 

Nonetheless, for the situation of interest in our study - the merger of binary black holes - it is currently unknown if there exists an electromagnetic counterpart \cite{perna2019limits}. However, an intriguing possibility arises if gravitational field lines were to reconnect at a stable Lagrange point of orbiting Kerr black holes. In such a scenario, any matter trapped at that Lagrange point - which would be subjected to high pressure - would briefly become liberated from the field, potentially resulting in the ejection of material in the form of a jet. This mechanism could give rise to an electromagnetic counterpart and offer plausible explanations for jet phenomena with currently unknown origins.

The possibility of electromagnetic counterparts to binary black hole mergers has been proposed, with potential signals already detected \cite{graham2020candidate}. Various theories have been put forth to explain the mechanisms behind such signals \cite{kelly2017prompt}. To the best of my knowledge, investigation into gravitational reconnection has not been undertaken. Nevertheless, recent studies on magnetic reconnection in the vicinity of black holes have emerged \cite{AsenjoFelipeA2017RMRi,comisso2021magnetic}, led by researchers such as Felipe A. Asenjo and Luca Comisso.

Thus far, the project has primarily focused on acquiring an understanding of the relevant background material. Subjects that have been studied (including references to some of the texts I have used) include general relativity \cite{hartle2003gravity,carroll2019spacetime}, high-energy astrophysics \cite{longair2010high}, and plasma physics \cite{choudhuri_1998}. Additionally, I have undertaken simpler simulations as part of practice exercises. For instance, a Python program was developed to simulate the trajectory of both massive and massless particles around a Kerr black hole, showcasing phenomena such as the innermost stable circular orbits and frame dragging.


\section{Research Methods and Plan}


My research will involve both theoretical calculations, along with computer simulations (numerical relativity). For the theoretical calculations, these will primarily be conducted via traditional 'pen and paper' methods, supplemented with computational tools such as Mathematica and general relativity packages like the one cited in reference  \cite{Shoshany2021_OGRe}. The computer simulations, if we decide to go down the numerical relativity route, will likely involve the \href{http://einsteintoolkit.org/about.html}{Einstein toolkit}, which has associated Python packages. The next step in the project is to begin simulating spacetime near a stable Lagrange point of orbiting black hole pairs.


\bibliographystyle{abbrv}
\bibliography{thesis_proposal/proposal_ref}


\end{document}