\documentclass{article}

% Language setting
% Replace `english' with e.g. `spanish' to change the document language
\usepackage[english]{babel}

% Set page size and margins
% Replace `letterpaper' with `a4paper' for UK/EU standard size
\usepackage[letterpaper,top=2cm,bottom=2cm,left=3cm,right=3cm,marginparwidth=1.75cm]{geometry}

% Useful packages
\usepackage{amsmath}
\usepackage{graphicx}
\usepackage[colorlinks=true, allcolors=blue]{hyperref}

\title{Gravitational field energy flow}
\author{Jordan Moncrieff}

\begin{document}
\maketitle

\begin{abstract}

\end{abstract}

\section{Poynting vectors}


\section{Flow of gravitational energy}

All fields for the fundamental forces of nature carry energy and momentum. While this is well established in electromagnetism via the energy momentum tensor, gravitational energy is much more subtle, with many believing that there is no gravitational energy momentum tensor. However, there are some work around in order to define the energy flow via the gravitational field.

The most basic way to analyze energy flow is with the electrogravitomagnetic formula for the Poynting vector. This method has an obvious flaw, the gravitational field is not defined by a vector field, it is defined by the rank 2 metric tensor field $g_{\mu \nu}(x)$. In the fully relativistic theory of gravity, the most famous field to describe energy flow is the Landau–Lifshitz pseudotensor.

\subsection{Landau–Lifshitz pseudotensor}

\subsection{Alternatives}

\section{Visualizing energy flow}

LIC method

\section{Application to BBHs}

\subsection{Does this suggest reconnection?}


\bibliographystyle{alpha}
\bibliography{sample}

\end{document}